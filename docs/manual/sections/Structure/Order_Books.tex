\section{Time-series Data Structure Layout}

\subsection{Introduction}


Since this is an order book \textit{recorder} my use until now has been to record the order book. However there are accessors in the \texttt{LOB\_funcs.py} file, under in the \texttt{history} class. In the /test folder there is a small use case called \texttt{csv\_out\_test.py} wiith the associated jupyter notebook pending.

1. For now we only have the full order book and we call that\texttt{recorder\_full.py}. 

2. We change the \texttt{settings} variable in the \texttt{CSV\_out\_test.py} file that has arguments for:

\begin{center}
	\begin{tabular}{|l|l|l|l|}
		\hline
		Parameter & Function Arg & Type/Format & Description \\
		\hline
		Recording Duration & \texttt{duration}& \texttt{int} & recording time in seconds\\
		Position Range & \texttt{position\_range} & \texttt{int} & ordinal distance from the best bid(ask)\\
		Currency Pair & \texttt{currency\_pair} & \texttt{str} & The currency pair from list of supported pairs\\
		\hline
	\end{tabular}
\end{center}

\footnote{\href{https://help.coinbase.com/en/pro/trading-and-funding/cryptocurrency-trading-pairs/locations-and-trading-pairs}{List of currency pairs supported by Coinbase}}

%| Parameter                | Function Arg  | Type |  Description | 
%|-------------------------|-------------------|------| -------|
%| Recording Duration | duration      | int  | recording time in seconds | 
%| Position Range     | position_range| int  | ordinal distance from the best bid(ask) |
%| Currency Pair      | currency_pair | str  | [List of currency pairs supported by Coinbase](https://help.coinbase.com/en/pro/trading-and-funding/cryptocurrency-trading-pairs/locations-and-trading-pairs) |

3. When you are ready, you can start the build. When it finishes you should get a message \texttt{Connection Closed} from CoPrA. And the files for the limit order book for each side should be created with a timestamp in the form YYYY-MM-DDTHH:MM:SS.ffffff. Currently it will output everything given in the table below, but I will update the function to only output select things for people.
\begin{center}
\begin{tabular}{|l|c|l|}
	\hline
	Filename & side & Time Series Description \\
	\hline 
	L2\_orderbook\_events\_ask+timestamp& ask& Order book events on the ask side\\
	L2\_orderbook\_events\_bid+timestamp&bid& Order book events on the bid side \\
	L2\_orderbook\_events\_signed+timestamp& both & Order book events on both sides,\\
	&& - sign for bid, +sign for ask\\
	L2\_orderbook\_ask\_volm+timestamp& ask &Volume snapshots of order book on the ask side \\
	L2\_orderbook\_bid\_volm+timestamp& bid &Volume snapshots of order book on the bid side \\
	L2\_orderbook\_signed\_volm+timestamp & both &Volume  snapshots of the signed order book,\\
	&&- sign for bid, + sign for ask\\
	L2\_orderbook\_ask\_volm+timestamp& ask &Price snapshots of order book on the ask side \\
	L2\_orderbook\_bid\_volm+timestamp& bid &Price snapshots of order book on the bid side \\
	L2\_orderbook\_signed\_volm+timestamp& both &Price snapshots of the signed order book,\\
	&& - sign for bid, +sign for ask\\
	\hline
\end{tabular}
\end{center}

\subsection{Understanding The Raw Order Book Data}

The Coinbase exchange operates using the double auction model, the \href{https://docs.pro.coinbase.com/}{Coinbase Pro API}, and by extension the \href{https://copra.readthedocs.io/en/latest/}{CoPrA API} makes it relatively easy to get still images of an instance of the orderbook as \href{(https://docs.pro.coinbase.com/#the-level2-channel)}{\texttt{snapshots}} and it sends updates in real time of the volume at a particular price level as \href{https://docs.pro.coinbase.com/#the-level2-channel}{\texttt{l2\_update}} messages. If you would like to know more O'Hara's book\cite{Ohara:MMTheory} does a great job introducing the double auction model for the purposes of defining the types of orders, and how they record events and make sense of them. 

\subsubsection{Order Book Snapshots}

Private data providers (e.g., \href{https://data.nasdaq.com/BookViewer.aspx}{NASDAQ - BookViewer},\href{https://lobsterdata.com/}{LOBSTER} provide reconstructions of order books. Coinbase updates the levels on the instance of a update message from the exchange of the order book. The format of the order book snapshot at time $t$ is a $2\times n$ array for $n$ price levels, where the first element is the price-level, and the second element is the volume at that price-level. The time series of order book snapshots is also an array with dimensions $2\times m$, for $m$ recorded timestamps, where the first element is the timestamp,$t\_m$, and the second element contains the entire $2\times n$ snapshot at $t_m$.

\cleardoublepage
Event recording are a time series of MO, LO, CO's as afforded from the \texttt{l2\_update} messages which are used to update the price, volume pair size at each price level. The format of the Event recorder is as follows:

\begin{center}
	
	\begin{tabular}{|l|l|l|}
		\hline
		Item & Description & format \\
		\hline
		Timestamp & Timestamp of when the event occurred &  YYYY-MM-DDTHH:MM:SS.ffffff\\
		order type & MO, LO, CO & \texttt{str} $\in \left\{\texttt{`market',`limit', `cancellation'}\right\}$\\
		price level & price of event occurrence in quote currency & \texttt{float64}\\
		event size & size of event in base currency  & \texttt{float64}\\
		position & signed position (-- for bids, + for asks) & \texttt{int}\\
		mid price & (best-ask + best-bid)/2 & \texttt{float64}\\
		spread & best-ask + best-bid & \texttt{float64}\\
		\hline
	\end{tabular}
\end{center}


%\tablefootnote{given constraints on the tick size of quote currency (e.g., USD =: \$0.01) we round the event size to the nearest decimal that would correspond to a move constrained by the minimum tick size.}

\subsubsection{Signed Order Book}
\paragraph{Signed Order Book Snapshot Prices}
The signed orderbook takes a different approach to position labelling so please keep that in mind. (note: I should  shift the position index to start at 1, for singe side order book snapshot time series). The signed orderbook snapshot is generated in a similar fashion with a volume, and price at each position. However, it uses the convention established in [3] for the signed order book. where positions on the bid are negative, with negative volume (XTC). I'll show the default setting that displays the 5 best bids and asks on each side.

\begin{center}
	\begin{tabular}{|l|l|}
		\hline
		Item & Description/Format \\
		\hline
		Timestamp & YYYY-MM-DDTHH:MM:SS.ffffff\\ 
		$-n$ & price quote at the $n^{\text{th}}$ best bid (i.e., worst bid) \\
		$-n+1$ & aggregate XTC limit buys at the second to worst bid \\
		$\ldots$ & \ldots \\
		$-2$ & aggregate XTC being bid for at the $2^{nd}$ best bid \\
		$-1$ & aggregate XTC limit being bid for at the best bed bid \\
		$1$ & aggregate XTC offered at the best ask\\
		$2$ & aggregate XTC offered at the $2^{\text{nd}}$ best ask\\
		\ldots & \ldots \\
		$n-1$ & aggregate XTC offered at the second worst ask \\
		$n$ & aggregate XTC offered at the worst ask \\
		\hline
	\end{tabular}
\end{center}


Similar to the single side implementation, there is an associated price quote  (e.g., USD per XTC) snapshot generated at each timepoint. The default format is given below:

\begin{center}
	\begin{tabular}{|l|l|}
		\hline
		Item & Description/Format \\
		\hline
		Timestamp & YYYY-MM-DDTHH:MM:SS.ffffff\\ $-n$ & price quote at the $n^{\text{th}}$ best bid (i.e., worst bid) \\
		$-n+1$ & price quote at the second to worst bid \\
		$\ldots$ & \ldots \\
		$-2$ & price quote at the $2^{nd}$ best bid \\
		$-1$ & price quote at the best bed bid \\
		$1$ & price quote at the best ask\\
		$2$ & price quote at the $2^{\text{nd}}$ best ask\\
		\ldots & \ldots \\
		$n-1$ & price quote at the second worst ask \\
		$n$ & price quote at the worst ask \\
		\hline
	\end{tabular}
\end{center}

\paragraph{Signed Events}\hfill\break
\noindent Signed event recordings follow the convention from \href{https://arxiv.org/pdf/1011.6402.pdf}{The Price impact of Orderbook events}, where positive order flow is due to MO's on the buy side, CO on the sell side, and LO on the buy side. Conversely,  negative order flow is due to MO's on the sell side, CO on the buy side, and LO on the buy side. The format is similar to the single side order book events time-series, but the order volume is signed based on the aforementioned construction. 
\begin{center}
	\begin{tabular}{|l|l|l|}
		\hline
		Item & Description & format \\
		\hline
		Timestamp & Timestamp of when the event occurred &  YYYY-MM-DDTHH:MM:SS.ffffff\\
		order type & MO, LO, CO & \texttt{str} $\in \left\{\texttt{`market',`limit', `cancellation'}\right\}$\\
		price level & price of event occurrence in quote currency & \texttt{float64}\\
		event size & size of event in base currency & \texttt{float64}\\
		position & signed position (-- for bids, + for asks) & \texttt{int}\\
		side & bid/ask side where the event occurs & \texttt{str} $\in \left\{\texttt{`buy',`sell'}\right\}$\\
		mid price & (best-ask + best-bid)/2 & \texttt{float64}\\
		spread & best-ask + best-bid & \texttt{float64}\\
		\hline
	\end{tabular}
\end{center}

%y\footnote{given constraints on the tick size of quote currency (e.g., USD =: \$0.01) we round the event size to the nearest decimal that would correspond to a move constrained by the minimum tick size.}

\newpage
\subsection{Final notes on Orderbook interpretation}

write about how the program object sees the orderbook with an example of code. 


\begin{tikzpicture}[node distance=2cm]
	\node (start) [startstop] {Start};
\end{tikzpicture}


\newpage