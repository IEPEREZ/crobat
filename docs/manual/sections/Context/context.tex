\section{Usage}

\textbf{Preface:} crobat is used with an open websocket connection to the Coinbase exchange. It uses CoPrA to parse JSON objects, and it records time series arrays of order book snapshots and events. \hfill\break

We demonstrate where crobat is used by beginning with CoPrA's \texttt{heartbeat.py} example:

\begin{lstlisting}[language=Python]
import asyncio

from copra.websocket import Channel, Client

loop = asyncio.get_event_loop()

ws = Client(loop, Channel('heartbeat', 'BTC-USD'))

try:
loop.run_forever()
except KeyboardInterrupt:
loop.run_until_complete(ws.close())
loop.close()
\end{lstlisting}

If we were to execute the code, we would run methods in the instance of the class \textcolor{olive}{\texttt{Client}} based on the arrival of messages on the heartbeat channel for the pair BTC-USD. crobat primarily inherits methods from this class and adds logic for managing incoming messages. Below we introduce our own version of this loop found in \texttt{CSV\_out.py}.

\begin{lstlisting}[language=Python]
	import recorder_full as rec
	import asyncio, time
	from datetime import datetime
	import copra.rest
	from copra.websocket import Channel, Client
	
	class input_args(object):
		def __init__(self, currency_pair='ETH-USD', position_range=5, recording_duration=5):
			self.currency_pair = currency_pair
			self.position_range = position_range
			self.recording_duration = recording_duration
	
	def main():
		settings_1 = input_args()
	
		loop = asyncio.get_event_loop()
	
		channel1 = Channel('level2', settings_1.currency_pair) 
		channel2 = Channel('ticker', settings_1.currency_pair)
	
		ws_1 = rec.L2_Update(loop, channel1, settings_1)
		ws_1.subscribe(channel2)
	
		timestart=datetime.utcnow()
	
	try:     
		loop.run_forever()
	except KeyboardInterrupt:
		loop.run_until_complete(ws_1.close())
		loop.close()
	
	if __name__ == '__main__':
		main()
\end{lstlisting}

In this modified event loop we have the context where the bulk of crobat sits, the class l2\_update. we can the \texttt{loop} object that is passed to \texttt{ws\_1}, the instance of the class \texttt{L2\_Update}, receives \texttt{msg} objects through the async methods inherited from Client. Here we give an example of what \texttt{l2\_Update} looks like:\newpage

\begin{lstlisting}[language=Python]
import asyncio, time
from datetime import datetime
import copra.rest
from copra.websocket import Channel, Client

class L2_Update(Client):
	def __init__(self, loop, channel, input_args):
	self.time_now = datetime.utcnow() #initial start time
	self.position_range = input_args.position_range
	self.recording_duration = input_args.recording_duration
	self.snap_received = False
	super().__init__(loop, channel)

	def on_open(self):
		print("Let's count the L2 messages!", self.time_now)
		super().on_open() # inheriting things from the parent class who really knows    

	def on_message(self, msg):
		if msg['type'] in ['snapshot']:
			print("received the snapshot")
			time = self.time_now
			self.snap_received = True
		if msg['type'] in ['ticker']:
			print("received ticker message")
			if self.snap_received:
				# Do Something Here
			else:
				print("market order arrived but no snapshot received yet")

		if msg['type'] in ['l2update']:
			print("received an l2update message")
		else:
			print("unknown message")

		if (datetime.utcnow() - self.time_now).total_seconds() > 		self.recording_duration:
			self.loop.create_task(self.close()) 

	def on_close(self, was_clean, code, reason):
		print("Connection to server is closed")
		print(was_clean)
		print(code)
		print(reason)
		
def main():
	pass

if __name__ == '__main__':
	main()
\end{lstlisting}

Here we can see that we can create an instance of this class, and on its initialization we create arrays that can hold our changes and order book states. The logic tree for when each type of message arrives can dictate what kinds of records, and changes to the snapshot are made the arrival of a message. In this current version, the history module contains methods to both update the order book and record changes, but in future versions the processes should be separated as other may have better ideas as to what to do with order book changes. In the following section we will introduce the modules and their associated functions and classes.