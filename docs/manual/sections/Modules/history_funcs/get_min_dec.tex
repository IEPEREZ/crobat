\paragraph{\textit{function} \textcolor{blue}{\texttt{get\_min\_dec}}\texttt{(min\_currency\_denom, min\_asset\_value)}}\hfill\break
\noindent \textbf{Description:} determines the decimal place for the smallest amount of base currency needed for the minimal denomination of the float currency. 

\noindent Example:  Suppose you have an exchange rate of $10010.99 USD/BTC$, (i.e., you need $10010.99 USD$ to buy $1 BTC$) then the amount of BTC in 1 cent or 0.01 USD which is the smallest denomination of the USD is $0.000000999 BTC$. We see that this is $10^{-7}$,  \textcolor{blue}{\texttt{get\_min\_dec}} would return $7$, to pass onto \texttt{numpy.around}.


\begin{tabular}{r r l }
	\textbf{parameters:}	& min\_currency\_denom : & float64\\
	&  & the minimum currency denomination (e.g., 0.01 USD)\\
	& min\_asset\_value:& float646\\
	&& the smallest current valuation of the asset.\\
\end{tabular}

\begin{tabular}{l c l}
	\textbf{returns:} & min\_dec\_out & int\\
	& & The number of decimal places out for the smallest movement of the asset for a minimal denomination ..sdfkljfsdlka rewwrite.  
\end{tabular}

\begin{tabular}{l c l}
	\textbf{latent changes:} & None: & None\\
\end{tabular}

\textbf{Example: None}
\begin{lstlisting}[language=Python]
??????


\end{lstlisting}