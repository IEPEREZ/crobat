\paragraph{\textit{function} \textcolor{blue}{\texttt{on\_message}}\texttt{(self, msg)},}
\hfill\break
\textbf{Description:} 
After matching \texttt{msg['type']} to \texttt{'snapshot','ticker', 'l2update'} do one of three actions,
\begin{center}
	\begin{tabular}{ |c|c| }
		\hline
		\texttt{msg[`type'] ==} & action\\
		\hline
		\texttt{`snapshot'}&initialize the limit order book using the \texttt{initialize\_snap\_events}\\
		\hline
		\texttt{`ticker'}&  parse a market order,\\
		\hline
		\texttt{`l2update'}& parse a limit order insertion or cancellation to the limit order book.\\
		\hline
	\end{tabular}
\end{center}

\noindent \textit{restated from} \texttt{CoPrA}:\hfill\break
\fbox{\begin{minipage}{\textwidth}
		\texttt{on\_message}on is called every time a message is received. message is a dict representing the message. Its content will depend on the type of message, the channels subscribed to, etc. Please read Coinbase Pro’s WebSocket API documentation to learn about these message formats.
		
		Note that with the exception of errors, every other message triggers this method including things like subscription confirmations. Your code should be prepared to handle unexpected messages.
		
		This default method just prints the message received. If you override this method, there is no need to call the parent method from your subclass’ method.
\end{minipage}}

\begin{tabular}{r r l }
	\textbf{parameters:}	& msg: & dict\\
\end{tabular}

\begin{tabular}{r r l}
	\textbf{returns:} & None: & None\\
\end{tabular}

\textbf{Example:}
\begin{lstlisting}[language=Python]]
# Let our module be imported and our channels be assigned as follows:
import recorder_full as rec
channel1= Channel('level2', 'BTC-USD')
channel2= Channel('ticker', 'BTC-USD')


#Let the instance of the class L2_Update be ws_1 
ws_1 = rec.L2_Update(loop, channel1, settings )
ws_1.subscribe(channel2)

#Let the message received from the websocket be:
msg = {
	"type": "l2update",
	"product_id": "BTC-USD",
	"time": "2019-08-14T20:42:27.265Z",
	"changes": [
	[
	"buy",
	"10101.80000000",
	"0.162567"
	]
	]
}

# let the current state of the orderbook be defined as follows:
ws_1.orderbook_instance.snapshot_bid = [
	[10101.00, 5.23], [10101.50, 1.11], [10101.80, 0.5]
	]

ws_1.orderbook_instance.snapshot_signed = [
	[10101.00, -5.23], [10101.50, -1.11], [10101.80, -0.5],
	[10101.90, 0.4], [10102.00, 1.3], [10102.10, 5.00]
	]

ws_1.orderbook_instance.bid_events = []
ws_1.orderbook_instance.signed_events = []
#Suppose on loop you this message was passed to on_message(msg)
on_message)(msg)

# the new values for the the snapshots, and events would be:
>>print(ws_1.orderbook_instance.snapshot_bid)
>>[[10101.00, 5.23], [10101.50, 1.11], [10101.80, 0.162567]]

>>print(ws_1.orderbook_instance.snapshot_signed)
>>[[10101.00, -5.23], [10101.50, -1.11], [10101.80, -0.162567], [10101.90, 0.4], [10102.00, 1.3], [10102.10, 5.00]]

>>print(ws_1.orderbook_instance.bid_events)
>>[[2019-08-14T20:42:27.265Z, 10101.80, cancellation, 0.337433, 1, 10101.85, 0.10]]

>>print(ws_1.orderbook_instance.signed_events)
>>[[2019-08-14T20:42:27.265Z, 10101.80, cancellation, 0.337433, -1, 10101.85, 0.10]]
\end{lstlisting}

