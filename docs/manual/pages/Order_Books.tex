\section{Time-series Data Structure Layout}

\subsection{Introduction}


Since this is an orderbook <u>recorder</u> my use until now has been to record the orderbook. However there are accessors in the ```LOB\_funcs.py``` file, under in the *history* class. In the /test folder there is a small usecase if you would like to see it but documentation is pending.

1. For now we only have the full orderbook, with no regard for ticksize, and we call that ```recorder\_full.py```. 

2. We change the ```settings``` variable in the ```CSV\_out\_test.py``` file that has arguments for:

\begin{center}
	\begin{tabular}{|l|l|l|l|}
		\hline
		Parameter & Function Arg & Type/Format & Description \\
		\hline
		Recording Duration & \texttt{duration}& \texttt{int} & recording time in seconds\\
		Position Range & \texttt{position\_range} & \texttt{int} & ordinal distance from the best bid(ask)\\
		Currency Pair\footnote{\href{https://help.coinbase.com/en/pro/trading-and-funding/cryptocurrency-trading-pairs/locations-and-trading-pairs}{List of currency pairs supported by Coinbase}} & \texttt{currency\_pair} & \texttt{str} &\\
		\hline
	\end{tabular}
	
\end{center}

%| Parameter                | Function Arg  | Type |  Description | 
%|-------------------------|-------------------|------| -------|
%| Recording Duration | duration      | int  | recording time in seconds | 
%| Position Range     | position_range| int  | ordinal distance from the best bid(ask) |
%| Currency Pair      | currency_pair | str  | [List of currency pairs supported by Coinbase](https://help.coinbase.com/en/pro/trading-and-funding/cryptocurrency-trading-pairs/locations-and-trading-pairs) |

3. When you are ready, you can start the build. When it finishes you should get a message ```Connection Closed``` from ```CoPrA```. And the files for the limit orderbook for each side should be created with a timestamp:

%|Filename|side|description|
%|----|----|----|
%|L2_orderbook_events_askYYYY-MM-DDTHH:MM:SS.ffffff| ask|Time series of order book events on the ask side|
%|L2_orderbook_events_bidYYYY-MM-DDTHH:MM:SS.ffffff| bid| Time series of order book events on the bid side |
%|L2_orderbook_events_signedYYYY-MM-DDTHH:MM:SS.ffffff| both | Time series of order book events on both sides, - sign for bid, +sign for ask|
%|L2_orderbook_ask_volmYYYY-MM-DDTHH:MM:SS.ffffff| ask |Time series of the volume snapshots of order book on the ask side |
%|L2_orderbook_bid_volmYYYY-MM-DDTHH:MM:SS.ffffff| bid |Time series of the volume snapshots of order book on the bid side |
%|L2_orderbook_signed_volmYYYY-MM-DDTHH:MM:SS.ffffff | both |Time series of the volume  snapshots of the signed order book, - for bid, + for ask|
%|L2_orderbook_ask_volmYYYY-MM-DDTHH:MM:SS.ffffff| ask |Time series of the price snapshots of order book on the ask side |
%|L2_orderbook_bid_volmYYYY-MM-DDTHH:MM:SS.ffffff| bid |Time series of the price snapshots of order book on the bid side |
%|L2_orderbook_signed_volmYYYY-MM-DDTHH:MM:SS.ffffff | both |Time series of the price snapshots of the signed order book, - for bid, + for ask|

\subsection{Understanding The Raw Order Book Data}

The coinbase exchange operates using the double auction model, the \href{https://docs.pro.coinbase.com/}{Coinbase Pro API}, and by extension the \href{https://copra.readthedocs.io/en/latest/}{CoPrA API} makes it relatively easy to get still images of an instance of the orderbook as \href{(https://docs.pro.coinbase.com/#the-level2-channel)}{\texttt{snapshots}} and it sends updates in real time of the volume at a particular price level as \href{https://docs.pro.coinbase.com/#the-level2-channel}{\texttt{l2\_update}} messages. If you would like to know more, the cited papers do a great job introducing the double auction model for the purposes of defining the types of orders, and how they record events and make sense of them. 

\subsubsection{Order Book Snapshots}

Below there is a graph of the snapshot where bids (green) show open limit orders to buy the 1 unit of the cryptocurrency below \$7085.930, and asks (red) show open limit orders to buy 1 unit above \$7085.930. The x-axis shows the price points, and the y-axis is the aggregate size at the price level. Note that  the signed order book calls volume on the bid side negative. 

%<img src="https://raw.githubusercontent.com/orderbooktools/crobat/master/images/figure_1.png" >

Early and current works relied on exchanges and private data providers (e.g., \href{https://data.nasdaq.com/BookViewer.aspx}{NASDAQ - BookViewer},\href{https://lobsterdata.com/}{LOBSTER} to provide reconstructions of order books. Earlier works were limited to taking snapshots and inferring the possible sequence of order book events between states. Coinbase and by extension crobat update the levels on the instance of a update message from the exchange so there is no guess as to what happened between states of the order book. The current format of the order book snapshot is not aggregated. The format of the order book snapshot for a single side is shown below

\begin{center}
	\begin{tabular}{|l|l|}
		\hline
		Item & Description/Format \\
		\hline
		Timestamp & YYYY-MM-DDTHH:MM:SS.ffffff\\ $1$ & total BTC at position 1 \\
		$2$ &total BTC at position 1 \\
		$\ldots$ & \ldots \\
		$n$ & total BTC at position n \\
		\hline
	\end{tabular}
\end{center}

\textbf{Incl. sample output of an entry}

The associated price quote (price quote (USD per XTC))snapshot is also generated, to make generation of market depth feasible. 

\begin{center}
	\begin{tabular}{|l|l|}
		\hline
		Item & Description/Format \\
		\hline
		Timestamp & YYYY-MM-DDTHH:MM:SS.ffffff\\ 
		$1$ & price quote at position 1 \\
		$2$ &price quote at position 1 \\
		$\ldots$ & \ldots \\
		$n$ & price quote at position n \\
		\hline
	\end{tabular}
\end{center}


Event recording are a timeseries of MO, LO, CO's as afforded from the \texttt{l2\_update} messages which are used to update the price, volume pair size at each price level. The format of the Event recorder is as follows:

\begin{center}
	
	\begin{tabular}{|l|l|l|}
		\hline
		Item & Description & format \\
		\hline
		Timestamp & Timestamp of when the event occurred &  YYYY-MM-DDTHH:MM:SS.ffffff\\
		order type & MO, LO, CO & \texttt{str} $\in \left\{\texttt{`market',`limit', `cancellation'}\right\}$\\
		price level & price of event occurrence in quote currency & \texttt{float64}\\
		event size & size of event in base currency\footnote{given constraints on the tick size of quote currency (e.g., USD =: \$0.01) we round the event size to the nearest decimal that would correspond to a move constrained by the minimum tick size.} & \texttt{float64}\\
		position & signed position (-- for bids, + for asks) & \texttt{int}\\
		mid price & (best-ask + best-bid)/2 & \texttt{float64}\\
		spread & best-ask + best-bid & \texttt{float64}\\
		\hline
	\end{tabular}
\end{center}

\subsubsection{Signed Order Book}
\paragraph{Signed Order Book Snapshot Prices}
The signed orderbook takes a different approach to position labelling so please keep that in mind. (note: I should  shift the position index to start at 1, for singe side order book snapshot time series). The signed orderbook snapshot is generated in a similar fashion with a volume, and price at each position. However, it uses the convention established in [3] for the signed order book. where positions on the bid are negative, with negative volume (XTC). I'll show the default setting that displays the 5 best bids and asks on each side.

\begin{center}
	\begin{tabular}{|l|l|}
		\hline
		Item & Description/Format \\
		\hline
		Timestamp & YYYY-MM-DDTHH:MM:SS.ffffff\\ 
		$-n$ & price quote at the $n^{\text{th}}$ best bid (i.e., worst bid) \\
		$-n+1$ & aggregate XTC limit buys at the second to worst bid \\
		$\ldots$ & \ldots \\
		$-2$ & aggregate XTC being bid for at the $2^{nd}$ best bid \\
		$-1$ & aggregate XTC limit being bid for at the best bed bid \\
		$1$ & aggregate XTC offered at the best ask\\
		$2$ & aggregate XTC offered at the $2^{\text{nd}}$ best ask\\
		\ldots & \ldots \\
		$n-1$ & aggregate XTC offered at the second worst ask \\
		$n$ & aggregate XTC offered at the worst ask \\
		\hline
	\end{tabular}
\end{center}


Similar to the single side implementation, there is an associated price quote  (e.g., USD per XTC) snapshot generated at each timepoint. The default format is given below:

\begin{center}
	\begin{tabular}{|l|l|}
		\hline
		Item & Description/Format \\
		\hline
		Timestamp & YYYY-MM-DDTHH:MM:SS.ffffff\\ $-n$ & price quote at the $n^{\text{th}}$ best bid (i.e., worst bid) \\
		$-n+1$ & price quote at the second to worst bid \\
		$\ldots$ & \ldots \\
		$-2$ & price quote at the $2^{nd}$ best bid \\
		$-1$ & price quote at the best bed bid \\
		$1$ & price quote at the best ask\\
		$2$ & price quote at the $2^{\text{nd}}$ best ask\\
		\ldots & \ldots \\
		$n-1$ & price quote at the second worst ask \\
		$n$ & price quote at the worst ask \\
		\hline
	\end{tabular}
\end{center}

\paragraph{Signed Events}

\medskip
\noindent Signed event recordings follow the convention from \href{https://arxiv.org/pdf/1011.6402.pdf}{The Price impact of Orderbook events}, where positive order flow is due to MO's on the buy side, CO on the sell side, and LO on the buy side. Conversely,  negative order flow is due to MO's on the sell side, CO on the buy side, and LO on the buy side. The format is similar to the single side order book events time-series, but the order volume is signed based on the aforementioned construction. 

\begin{center}
	
	\begin{tabular}{|l|l|l|}
		\hline
		Item & Description & format \\
		\hline
		Timestamp & Timestamp of when the event occurred &  YYYY-MM-DDTHH:MM:SS.ffffff\\
		order type & MO, LO, CO & \texttt{str} $\in \left\{\texttt{`market',`limit', `cancellation'}\right\}$\\
		price level & price of event occurrence in quote currency & \texttt{float64}\\
		event size & size of event in base currency\footnote{given constraints on the tick size of quote currency (e.g., USD =: \$0.01) we round the event size to the nearest decimal that would correspond to a move constrained by the minimum tick size.} & \texttt{float64}\\
		position & signed position (-- for bids, + for asks) & \texttt{int}\\
		side & bid/ask side where the event occurs & \texttt{str} $\in \left\{\texttt{`buy',`sell'}\right\}$\\
		mid price & (best-ask + best-bid)/2 & \texttt{float64}\\
		spread & best-ask + best-bid & \texttt{float64}\\
		\hline
	\end{tabular}
	
\end{center}

\subsection{Final notes on Orderbook interpretation}

write about how the program object sees the orderbook with an example of code. 


\begin{tikzpicture}[node distance=2cm]
	\node (start) [startstop] {Start};
\end{tikzpicture}


\newpage