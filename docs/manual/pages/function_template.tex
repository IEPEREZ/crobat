%%%%%%%%%%%%%%%%%%%%%%%%%%%%%%%%%%%%%%%%%
%%                                     %%
%% Function Documentation Template     %%
%%									   %%
%%%%%%%%%%%%%%%%%%%%%%%%%%%%%%%%%%%%%%%%%


%%% DESCRIPTION %%%%
%%% Template to create function documentation in latex
%%% Follows a similar style to the documentation style in the python package Pandas 

\paragraph{\textit{function} \textcolor{blue}{\texttt{\_\_my\_function\_\_}}\texttt{(self, param\_1, param\_2=False, param\_3)}}\hfill\break
\noindent \textbf{Description:} Put your description here. In the next table we will state the parameters, on the left, with the object type on the right. just below it on the next line, we will give a brief description of the input parameter. After that we will use the returns portion in the exact same way. The final section is an example that uses \texttt{lstlisting} package to write snippets of python code for us. The filled out template is then included in the class .tex file and the class .tex file is included in the master manual .tex file.

For example this function will convert \texttt{param\_1} as a float to the nearest significant digits as defined by \texttt{param\_3} and will multiply by -1 if \texttt{param\_2:=True}. 

\begin{tabular}{r r l }
	\textbf{parameters:}	& param\_1: & float64\\
	&  & some input that will be passed to the function\\
	& param\_2:& boolean\\
	&& a conditional that will be checked by the function\\
	& param\_3:& int\\
	&& some int that is needed for the function	
\end{tabular}

\begin{tabular}{l c l}
	\textbf{returns:} & trans\_number & float64\\
	& & The number with the correct significant digits and sign. 
\end{tabular}

\begin{tabular}{l c l}
	\textbf{latent changes:} & trans\_number & float64\\
	& & The number with the correct significant digits and sign. 
\end{tabular}

\textbf{Example: None}
	\begin{lstlisting}[language=Python]
	 x = __my_function__(1.234, True, 3)
	 >>print(x)
	 >>-1.23
	 
	 x = __my_function__(1.234, False, 3)
	 >>print(x)
	 >>1.23
	 \end{lstlisting}